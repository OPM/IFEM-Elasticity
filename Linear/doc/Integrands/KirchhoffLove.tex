% $Id$
%%%%%%%%%%%%%%%%%%%%%%%%%%%%%%%%%%%%%%%%%%%%%%%%%%%%%%%%%%%%%%%%%%%%%%%%%%%%%%%%
\documentclass{article}
\usepackage{eqalign}
\topmargin    -13mm
\oddsidemargin  0mm
\evensidemargin 0mm
\textheight   247mm
\textwidth    160mm
\def\R{{\rm I\mkern-4muR}}
\def\deriv#1#2{\frac{\partial#1}{\partial#2}}
\def\dderiv#1#2{\frac{\partial^2#1}{\partial#2^2}}
\def\dxderiv#1#2#3{\frac{\partial^2#1}{\partial#2\partial#3}}
\def\dV{\:{\rm d}A}
\def\dA{\:{\rm d}S}
%%%%%%%%%%%%%%%%%%%%%%%%%%%%%%%%%%%%%%%%%%%%%%%%%%%%%%%%%%%%%%%%%%%%%%%%%%%%%%%%
\title{The Kirchhoff-Love thin plate problem}
\author{\sl Knut Morten Okstad, SINTEF Digital}
\begin{document}
\maketitle
%%%%%%%%%%%%%%%%%%%%%%%%%%%%%%%%%%%%%%%%%%%%%%%%%%%%%%%%%%%%%%%%%%%%%%%%%%%%%%%%

\section{Introduction}

This document describes the strong- and associated weak form of the linear
thin plate problem based on Kirchhoff-Love plate theory, which is included
in the linear elasticity application of {\sl IFEM}.
Refer to the class {\tt KirchhoffLovePlate} of the source code for the actual
integrand implementation.

\section{Strong form}

Given a distributed transverse load $p(x,y)$
defined over a domain $\Omega\subset\R^2$,
a bending moment $\bar{M}(x,y)$ and twist moment $\bar{T}(x,y)$
defined over the boundary $\partial\Omega_m$,
a transverse shear force $\bar{Q}(x,y)$
defined over the boundary $\partial\Omega_q$, and
two functions $\bar{w}(x,y)$ and $\bar{\theta}(x,y)$ defined over the boundaries
$\partial\Omega_w=\partial\Omega\setminus\partial\Omega_q$ and
$\partial\Omega_\theta=\partial\Omega\setminus\partial\Omega_m$, respectively,
find the scalar function $w(x,y)\in{\cal W}(\Omega)$ satisfying
%
\begin{eqnarray}
  \label{eq:strong}
  \left.\eqalign{
    \dderiv{m_{xx}}{x} + 2\dxderiv{m_{xy}}{x}{y} + \dderiv{m_{yy}}{y} &\;=\;-p \cr
    m_{xx} &\;=\; -D\left(\dderiv{w}{x} + \nu\dderiv{w}{y}\right) \cr
    m_{yy} &\;=\; -D\left(\dderiv{w}{y} + \nu\dderiv{w}{x}\right) \cr
    m_{xy} &\;=\; -D(1-\nu)\dxderiv{w}{x}{y}}
  \right\} &\forall& \{x,y\}\in\overline{\Omega} \\[2mm]
  \label{eq:neumannI}
  \left.\eqalign{
    m_{xx}n_x + m_{xy}n_y &\;=\; \bar{M} \cr
    m_{xy}n_x + m_{yy}n_y &\;=\; \bar{T}}
  \right\} &\forall& \{x,y\}\in\partial\Omega_m \\[2mm]
  \label{eq:neumannII}
  \left(\deriv{m_{xx}}{x} + \deriv{m_{xy}}{y}\right)n_x +
  \left(\deriv{m_{xy}}{x} + \deriv{m_{yy}}{y}\right)n_y \;=\;
  \bar{Q} &\forall& \{x,y\}\in\partial\Omega_q \\[2mm]
  w \;\;=\; \bar{w} &\forall& \{x,y\}\in\partial\Omega_w \\
  \deriv{w}{x}n_x + \deriv{w}{y}n_y \;=\;\; \bar{\theta}
  &\forall& \{x,y\}\in\partial\Omega_\theta
\end{eqnarray}
%
where $D=\frac{Et^3}{12(1-\nu^2)}$ is the plate stiffness parameter composed of
the Young's modulus $E$, the Poisson's ratio $\nu$, and the plate thickness $t$.
The two terms inside the parentheses of Equation~(\ref{eq:neumannII}) equals the
transverse shear forces, $q_x$ and $q_y$, respectively, and $n_x$ and $n_y$ are
the components of the outward-directed unit normal vector on $\partial\Omega$.
The solution space ${\cal W}(\Omega)$ defines the set of all admissible solution
functions $w$ on the domain $\Omega$, with the additional constraints
$w=\bar{w}(x,y)\;\forall\{x,y\}\in\partial\Omega_w$ and
$\deriv{w}{x}n_x+\deriv{w}{y}n_y=\bar{\theta}(x,y)
\;\forall\{x,y\}\in\partial\Omega_\theta$.
Assuming the plate stiffness $D$ is constant, the Equations~(\ref{eq:strong})
can be combined into the following fourth-order partial differential equation
%
\begin{equation}
  \label{eq:strong-constD}
  \frac{\partial^4w}{\partial x^4} +
 2\frac{\partial^4w}{\partial x^2\partial y^2} +
  \frac{\partial^4w}{\partial y^4} = \frac{p}{D}
  \quad\forall\quad\{x,y\}\in\overline{\Omega}
\end{equation}
%
Similarly, Equation~(\ref{eq:neumannII}) can be transformed by substituting the
expressions for $m_{xx}$, $m_{yy}$ and $m_{xy}$ from Equation~(\ref{eq:strong}),
resulting in
%
\begin{equation}
  \label{eq:neumann-shear}
  \deriv{}{x}\left(\dderiv{w}{x} + \dderiv{w}{y}\right)n_x +
  \deriv{}{y}\left(\dderiv{w}{x} + \dderiv{w}{y}\right)n_y = -\frac{\bar{Q}}{D}
  \quad\forall\quad\{x,y\}\in\partial\Omega_q
\end{equation}

\section{Weak form}

To develop the weak form, it is convenient to introduce the shortened notation
for partial differentiation $(\cdot)_{,\alpha}:=\deriv{(\cdot)}{x_\alpha}$, where
$\alpha$ is a running index over the coordinate directions ($\alpha=1\ldots2$),
and assuming Einsteins summation convention over repetitive indices,
e.g., $a_{\alpha\alpha} := a_{11} + a_{22}$.
With $\bar{M}_1:=\bar{M}$ and $\bar{M}_2:=\bar{T}$,
the Equations~(\ref{eq:strong-constD}), (\ref{eq:neumannI})
and~(\ref{eq:neumann-shear}), respectively,
can then be written in the following compact form

%
\begin{eqnarray}
  \label{eq:strong-omega}
  D\,w_{,\alpha\alpha\beta\beta} \;=\; p \quad
  &\forall&x_\alpha\in\overline{\Omega} \\[2mm]
  \label{eq:neumann-m}
  m_{\alpha\beta}n_\beta \;=\; \bar{M}_\alpha
  &\forall&x_\alpha\in\partial\Omega_m \\[2mm]
  \label{eq:neumann-q}
  D\,w_{,\alpha\alpha\beta}n_\beta \;=\: -\bar{Q}
  &\forall&x_\alpha\in\partial\Omega_q
\end{eqnarray}

The weak form is obtained by multiplying Equation~(\ref{eq:strong-omega}) by
a test function $v(x_\alpha)\in{\cal V}(x_\alpha)$ and then integrating over
the domain $\Omega$, viz.
%
\begin{equation}
  \label{eq:tested}
  D\int_\Omega w_{,\alpha\alpha\beta\beta}\,v\dV = \int_\Omega p\,v\dV
\end{equation}
%
The test space ${\cal V}(x_\alpha)$ is the same as ${\cal W}(x_\alpha)$,
except that their functions $v(x_\alpha)$ have the constraint
$v=0\;\forall\;x_\alpha\in\Omega_w$ and
$v_{,\alpha}n_\alpha=0\;\forall\;x_\alpha\in\partial\Omega_\theta$
instead of those of ${\cal W}(x_\alpha)$.

By applying the Green's identity (integration by parts),
Equation~(\ref{eq:tested}) is transformed to
%
\begin{equation}
  -D\int\limits_\Omega w_{,\alpha\alpha\beta}\,v_{,\beta}\dV +
   D\int\limits_{\partial\Omega} w_{,\alpha\alpha\beta}\,n_\beta\,v\dA =
   \int\limits_\Omega p\,v\dV
\end{equation}
%
The boundary integral of the second term can be further transformed by using
that $v(x_\alpha)=0\;\forall\;x_\alpha\in\Omega_q$ and substituting
Equation~(\ref{eq:neumann-q}), resulting in
%
\begin{equation}
  -D\int\limits_\Omega w_{,\alpha\alpha\beta}\,v_{,\beta}\dV =
  \int\limits_\Omega p\,v\dV +
  \int\limits_{\partial\Omega_q} \bar{Q}\,v\dA
\end{equation}
%
The left-hand-side term is next transformed by applying Green's identity a
a second time, resulting in
%
\begin{equation}
  D\int\limits_\Omega w_{,\alpha\beta}\,v_{,\alpha\beta}\dV -
  D\int\limits_{\partial\Omega} w_{,\alpha\beta}\,n_\alpha\,v_{,\beta}\dA =
  \int\limits_\Omega p\,v\dV +
  \int\limits_{\partial\Omega_q} \bar{Q}\,v\dA
\end{equation}
%
or simply
%
\begin{equation}
  a(w,v) \;=\; l(v)
\end{equation}
%
where we introduce the bilinear form $a(w,v)$
and the linear functional $l(v)$ as
%
\begin{eqnarray}
  \label{eq:bilinear form}
  a(w,v) &:=& D\int\limits_\Omega w_{,\alpha\beta}\,v_{,\alpha\beta}\dV
         \;=\;-\int\limits_\Omega m_{\alpha\beta}\,v_{,\alpha\beta}\dV \\
  l(v)   &:=& \int\limits_\Omega p\,v\dV +
              \int\limits_{\partial\Omega_q} \bar{Q}\,v\dA +
              \int\limits_{\partial\Omega_m} \bar{M}_\alpha\,v_{,\alpha}\dA
\end{eqnarray}

\section{Energy norms}

The computed finite element (FE) solution can be assessed by evaluating norm.
For a given FE solution $w^h$, we therefore define its energy norm as
%
\begin{equation}
  \label{eq:internal energy}
  U^h = \|w^h\|_E := \sqrt{a(w^h,w^h)}
\end{equation}
%
and the corresponding external energy is
%
\begin{equation}
  U_{ext}^h = \sqrt{l(w^h)}
\end{equation}
%
The FE implementation can therefore be verified by always asserting that
$U^h=U_{ext}^h$ for any problem setup.
Notice that in the evaluation of Equation~(\ref{eq:internal energy}),
we do not use the primary solution, $w^h$,
but only the secondary solution, $m_{\alpha\beta}^h$,
through Equation~(\ref{eq:bilinear form}).

\section{Error estimates}

An estimate of the discretization error in the FE solution can be obtained by
projecting the discontinuous secondary solution, $m_{\alpha\beta}^h$ onto a
continuous basis of higher order (or same order as the primary solution, $w^h$),
resulting in the recovered solution, $m_{\alpha\beta}^*$.
In {\sl IFEM}, several projection methods for obtaining $m_{\alpha\beta}^*$ are
available, e.g., discrete and continuous global $L_2$-projection.

With $m_{\alpha\beta}^*$ available, the discretization error is estimated by
%
\begin{equation}
  \label{eq:error estimate}
  \eta^{RES} = \|w^*-w^h\|_E +
  \sqrt{\sum_{k=1}^{n_{\rm el}}\left\{h_k^2\|R^*\|_{L_2(\Omega_k)}^2 +
    \frac{h_k}{2}\left(\|J_m^*\|_{L_2(\partial{\Omega_m}_k)}^2 +
                       \|J_q^*\|_{L_2(\partial{\Omega_q}_k)}^2\right)\right\}}
\end{equation}
%
where $h_k$ denotes some characteristic size of element $k$ (typically the
length of the longest diagonal). $R^*$, $J_m^*$, $J_q^*$ are, respectively,
the interior residual of the recovered solution $m_{\alpha\beta}^*$,
and the jump (boundary residual) between the recovered solution and the
prescribed edge moment $\bar{M}_\alpha$ and shear force $\bar{Q}$.
The notation $\|\cdot\|_{L_2(\Omega_k)}$ denotes the $L_2$-norm of the quantity
$(\cdot)$ over the sub-domain $\Omega_k$ of element $k$, i.e.
%
\begin{equation}
  \|\cdot\|_{L_2(\Omega_k)}^2 := \int\limits_{\Omega_k} (\cdot)^2\dV
  \quad\mbox{and}\quad
  \|\cdot\|_{L_2(\partial\Omega_k)}^2 := \int\limits_{\partial\Omega_k} (\cdot)^2\dA
\end{equation}

The interior residual is computed by inserting $m_{\alpha\beta}^*$ into
Equation~(\ref{eq:strong})$_1$:
%
\begin{equation}
  R^* = m_{\alpha\beta,\alpha\beta}^* - p
\end{equation}
%
The jump terms are computed from the Neumann boundary conditions,
Equations~(\ref{eq:neumann-m}) and~(\ref{eq:neumann-q}), respectively:

\end{document}
